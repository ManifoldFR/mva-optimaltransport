\documentclass{article}

\usepackage{subfiles}
\usepackage[a4paper,hmargin=2.6cm,vmargin=3.6cm]{geometry}
\usepackage{subcaption}
\usepackage{mathtools}
\usepackage{amssymb}
\usepackage{dsfont,mathrsfs}
\usepackage[dvipsnames]{xcolor}

\usepackage[
ruled,vlined,
linesnumbered
]{algorithm2e}

\usepackage{amsthm}
\usepackage[
framemethod=TikZ
]{mdframed}

\usepackage{verbatim}

\usepackage{hyperref,cleveref}
\usepackage{graphicx}
\usepackage{enumitem}

\setlist{itemsep=0pt,topsep=0pt}

\usepackage{csquotes}
\usepackage[
sorting=none,
minnames=1,
maxcitenames=2,
backend=biber
]{biblatex}

\addbibresource{../bibliography/references.bib}

%% Hyperref %%

\hypersetup{
	colorlinks,
	citecolor=Green
}

\crefalias{prop}{proposition}

%%% DEFINE MACROS %%%

%% Math %%

\newcommand{\RR}{\mathbb{R}}
\newcommand{\TT}{\mathbb{T}}
\newcommand{\QQ}{\mathbb{Q}}
\newcommand{\NN}{\mathbb{N}}
\newcommand{\BB}{\mathbb{B}}
\newcommand{\WW}{\mathbb{W}}
\newcommand{\EE}{\mathbb{E}}
\newcommand{\PP}{\mathbb{P}}

\newcommand{\bfR}{\mathbf{R}}
\newcommand{\bfP}{\mathbf{P}}


\newcommand{\calC}{\mathcal{C}}
\newcommand{\calE}{\mathcal{E}}
\newcommand{\calI}{\mathcal{I}}
\newcommand{\calH}{\mathcal{H}}
\newcommand{\calK}{\mathcal{K}}
\newcommand{\calL}{\mathcal{L}}
\newcommand{\calP}{\mathcal{P}}
\newcommand{\calO}{\mathcal{O}}
\newcommand{\calT}{\mathcal{T}}
\newcommand{\calS}{\mathcal{S}}
\newcommand{\calM}{\mathcal{M}}
\newcommand{\calW}{\mathcal{W}}
\newcommand{\calX}{\mathcal{X}}

\newcommand{\suchthat}{\mathrm{s.t.}}

\renewcommand{\phi}{\varphi}
\renewcommand{\epsilon}{\varepsilon}

\DeclareMathOperator{\divg}{div}
\DeclareMathOperator{\Ent}{Ent}
\DeclareMathOperator{\supp}{supp}
\DeclareMathOperator*{\argmin}{argmin}
\DeclareMathOperator*{\argmax}{argmax}

\DeclareMathOperator{\KL}{KL}
\DeclareMathOperator{\proj}{proj}
\DeclareMathOperator{\prox}{prox}
\DeclareMathOperator{\diag}{diag}

\numberwithin{equation}{section}

%% Colors %%

\colorlet{lightblue}{RoyalBlue!13!white}
\colorlet{midblue}{RoyalBlue!70}
\colorlet{midgreen}{OliveGreen!65}
\colorlet{darkred}{Red!90!Black}

\newcommand{\redfont}{\color{darkred}}
\newcommand{\bluefont}{\color{RoyalBlue}}
\newcommand{\greenfont}{\color{Green!90!black}}

%% THEOREM ENVS %%

\mdfsetup{
	outerlinewidth=1pt,
	innertopmargin=0cm,
}

%\newtheorem{prop}{Proposition}
\newmdtheoremenv[
hidealllines=true,
leftline=true,
linecolor=midblue,
backgroundcolor=RoyalBlue!3,
]{prop}{Proposition}


\theoremstyle{definition}
\newmdtheoremenv[
hidealllines=true,
leftline=true,
linecolor=Red!70,
]{remark}{Remark}

%% TITLE, AUTHOR %%

\author{
	Wilson Jallet\\
	\textit{École polytechnique}
}
\title{
	{\Large\sffamily Computational Optimal Transport}\\
	Numerical Tours}

\begin{document}
\maketitle

\section{Linear Programming}
\subsection{OT of distributions}

We use the following dataset (a circle inscribed within three arcs of another circle):
\begin{figure}[!h]
	\centering
	\includegraphics[width=.4\linewidth]{tp1/output_15_0.png}
	\caption{}
\end{figure}

This yields the transport plan in \cref{fig:output280}, as well as the connection defined by the optimal coupling \cref{fig:output300}.
\begin{figure}[!h]
	\centering
	\begin{subfigure}{.44\linewidth}
		\centering
		\includegraphics[width=.48\linewidth]{tp1/output_28_0}
		\caption{}
		\label{fig:output280}
	\end{subfigure}
	\begin{subfigure}{.44\linewidth}
		\centering
		\includegraphics[width=.48\linewidth]{tp1/output_30_0.png}
		\caption{}\label{fig:output300}
	\end{subfigure}
	\caption{}
\end{figure}


\subsection{Displacement interpolation}

Given the optimal transport plan $\bfP^\star$ we get the $W_2$-geodesic path $(\mu_t)_t$ as
\[
	\mu_t = \sum_{i,j} \bfP^\star_{i,j} \delta_{(1-t)x_i + ty_j}
\]
Applied on our dataset, we get the following barycenter distributions \cref{fig:output350}:
\begin{figure}[h!]
	\centering
	\includegraphics[width=0.8\linewidth]{tp1/output_35_0}
	\caption{}
	\label{fig:output350}
\end{figure}


\subsection{Optimal Assignment}

We adapt our previous dataset by adding a few points inside of the inner blue circle. In this case the OT plan is a permutation matrix $\bfP^\star = P_{\sigma^\star}$ and the heatmap of the matrix \cref{fig:output510} has two colors.
\begin{figure}[!h]
	\begin{subfigure}[t]{.33\linewidth}
		\includegraphics[width=\linewidth]{tp1/output_47_0.png}
		\caption{Our dataset.}
	\end{subfigure}~
	\begin{subfigure}[t]{.33\linewidth}
		\includegraphics[width=\linewidth]{tp1/output_51_0.png}
		\caption{Optimal transport plan $\bfP^\star$.}\label{fig:output510}
	\end{subfigure}~
	\begin{subfigure}[t]{.33\linewidth}
		\includegraphics[width=\linewidth]{tp1/output_53_0.png}
		\caption{Optimal Assignment plan $x_i = y_{\sigma^\star(j)}$.}\label{fig:output530}
	\end{subfigure}
	\caption{}\label{fig:OptimalAssignment}
\end{figure}


\section{Entropic Regularization of OT}

\subsection{Transport Between Point Clouds}

\paragraph{Exercise 1.} We use the following point cloud data:
\begin{figure}[h!]
	\centering
	\includegraphics[width=0.5\linewidth]{tp2/output_12_0}
	\caption{}
	\label{fig:output120}
\end{figure}\\
The optimal transport plan should have interesting structure: we see that in the results.
\begin{figure}[!h]
	\centering
	\begin{subfigure}{.49\linewidth}
		\centering
		\includegraphics[width=.8\linewidth]{tp2/output_33_1.png}\caption{Convergence of the $L^1$ error on the marginals.}
	\end{subfigure}
	\begin{subfigure}{.49\linewidth}
		\centering
		\includegraphics[width=.8\linewidth]{tp2/output_37_1.png}\caption{Optimal transport plan.}
	\end{subfigure}
	\caption{Convergence and optimal transport plan for regularization stength $\epsilon = 0.01$.}
\end{figure}

It holds that
\[
	P^{(l+1)}\mathds{1} = \diag{u^{(l+1)}}Kv^{(l+1)}
	= \frac{a}{Kv^{()}}\odot Kv^{(l+1)}
\]
which allows to compute the $L^1$ error efficiently.

\paragraph{Exercise 2.} We compute the regularized OT for different values of $\epsilon \in \{0.002, 0.004, 0.005, 0.01, 0.1, 0.3\}$.
\begin{figure}[!h]
	\includegraphics[width=\linewidth]{tp2/output_41_0.png}\caption{OT plans for varying $\epsilon$.}
\end{figure}
The smallest tested value for which there was no underflow is $\epsilon_{\min} = 4.10^{-3}$.
\begin{figure}[!h]
	\begin{subfigure}{.49\linewidth}
		\centering
		\includegraphics[width=0.7\linewidth]{tp2/output_44_1}
		\caption{Optimal transport plan $\bfP^\star_\epsilon$.}
		\label{fig:output441}
	\end{subfigure}
	\begin{subfigure}{.49\linewidth}
		\centering
		\includegraphics[width=0.7\linewidth]{tp2/output_45_0}
		\caption{Point cloud connections for $\epsilon = 4.10^{-3}$.}
		\label{fig:output450}
	\end{subfigure}
	\caption{Transport plan and connections.}
\end{figure}


\subsection{Transport Between Histograms}

We use the following data: the marginals are a mixture of Gaussians and a mixture of Laplace distributions, displayed \cref{fig:tp2output550}. Explicitly, the unnormalized functions are
\begin{equation}
\begin{aligned}
	f_1(x) &= \exp\left(-\frac{(x-1/2)^2}{2\sigma^2}\right) + 0.6\exp\left(-\frac{(x-0.65)^2}{2\sigma^2}\right)  \\
	f_2(x) &= 0.4 \exp\left(-\frac{|x-0.2|}\beta\right) + 0.6\exp\left(-\frac{|x-0.8|}\beta\right)
\end{aligned}
\end{equation}
and for our data we use $\sigma = 0.06$, $\beta = 0.04$.


\begin{figure}[!h]
	\centering
	\includegraphics[width=.9\linewidth]{tp2/output_55_0.png}\caption{Histograms of the marginal distributions.}\label{fig:tp2output550}
\end{figure}

\paragraph{Exercise 3.} We use the regularization strength $\epsilon = (0.03)^2$. The optimal coupling and transport map are given \cref{fig:tp2output690,fig:tp2output730}.
\begin{figure}[!h]
	\begin{subfigure}{.49\linewidth}
		\centering
		\includegraphics[width=.8\linewidth]{tp2/output_69_0.png}
		\caption{Optimal coupling between the histograms.}
		\label{fig:tp2output690}	
	\end{subfigure}~
	\begin{subfigure}{.49\linewidth}
		\centering
		\includegraphics[width=.8\linewidth]{tp2/output_73_0.png}
		\caption{Associated transport map.}
		\label{fig:tp2output730}	
	\end{subfigure}
	\caption{Coupling and transport map for $\epsilon = (0.03)^2$.}\label{fig:HistoCoupling}
\end{figure}

\paragraph{Bonus exercise.} \Cref{fig:HistoCouplingLowerReg} shows the coupling and transport map for a lower regularization strength $\epsilon$.
\begin{figure}[!h]
	\begin{subfigure}[t]{.49\linewidth}
		\centering
		\includegraphics[width=.8\linewidth]{tp2/output_79_0.png}
		\caption{Optimal coupling between the histograms.}
		\label{fig:tp2output790}	
	\end{subfigure}~
	\begin{subfigure}[t]{.49\linewidth}
		\centering
		\includegraphics[width=.8\linewidth]{tp2/output_80_0.png}
		\caption{Associated transport map.}
		\label{fig:tp2output800}	
	\end{subfigure}
	\caption{Coupling and transport map for $\epsilon = (0.03)^2$.}\label{fig:HistoCouplingLowerReg}
\end{figure}



\subsection{Wasserstein barycenters}

We use the data \cref{fig:tp2output1050} and regularization parameter $\epsilon = (0.04)^2$.
\begin{figure}[!h]
	\centering
	\includegraphics[width=.4\linewidth]{tp2/output_105_0.png}\caption{Bitmap image data.}
	\label{fig:tp2output1050}
\end{figure}

\paragraph{Exercise 4.} \Cref{fig:BregmanAlgo} shows the result.
\begin{figure}[!h]
	\centering
	\begin{subfigure}{.49\linewidth}
		\centering
		\includegraphics[width=0.9\linewidth]{tp2/output_129_0}
		\caption{}
		\label{fig:output1290}
	\end{subfigure}
	\begin{subfigure}{.49\linewidth}
		\centering
		\includegraphics[width=0.9\linewidth]{tp2/output_131_0}
		\caption{}
		\label{fig:output1310}
	\end{subfigure}
	\caption{Convergence and result of the Bregman algorithm to compute the Wasserstein barycenters of the images in \cref{fig:tp2output1050}.}\label{fig:BregmanAlgo}
\end{figure}

\paragraph{Exercise 5.} We compute Wasserstein barycenters for bilinear interpolation weights $t,s\in \{0, 0.25, 0.5, 0.75, 1.0\}$ -- see \cref{fig:BarycenterMosaic}.
\begin{figure}[!h]
	\centering
	\includegraphics[width=\linewidth]{tp2/output_136_0.png}\caption{Set of Wasserstein barycenters.}
	\label{fig:BarycenterMosaic}
\end{figure}


\clearpage

\section{Advanced Topics on Sinkhorn}

\subsection{Log-domain Sinkhorn}

\paragraph{Exercise 1.} We use the dataset defined in \cref{fig:tp3output80}.
\begin{figure}[!h]
	\centering
	\begin{subfigure}[t]{.49\linewidth}
		\centering
		\includegraphics[width=0.7\linewidth]{tp3/output_8_0}
		\caption{Point cloud data.}\label{fig:tp3output80}
	\end{subfigure}~
	\begin{subfigure}[t]{.49\linewidth}
		\centering
		\includegraphics[width=0.8\linewidth]{tp3/output_19_0.png}
		\caption{Convergence of the log-domain Sinkhorn for regularization strength $\epsilon = 10^{-2}$.}
		\label{fig:tp3output190}
	\end{subfigure}
	\caption{}
\end{figure}

\paragraph{Exercise 2.} We study the convergence of the Sinkhorn algorithm in log-domain for regularization parameters $\epsilon \in \{0.1, 0.05, 0.01, 0.01\}$. See \cref{fig:tp3output210}. Lowering regularization values can decrease convergence speed significantly.
\begin{figure}[!h]
	\centering
	\includegraphics[width=0.5\linewidth]{tp3/output_21_0.png}
	\caption{Convergence of log-domain Sinkhorn for multiple regularization parameters.}
	\label{fig:tp3output210}
\end{figure}



\subsection{Wasserstein Flows for Matching}

\begin{figure}[!h]
	\begin{subfigure}{.4\linewidth}
		\centering
		\includegraphics[width=\linewidth]{tp3/output_30_0}
		\caption{Lagrangian gradients of the energy functional $\calE(z)$ at the distribution $z$.}
		\label{fig:tp3output300}
	\end{subfigure}
	\begin{subfigure}{.59\linewidth}
		\centering
		\includegraphics[width=\linewidth]{tp3/output_36_0}
		\caption{Evolution of the iterate distribution $z^{(t)}$ along the gradient steps.}
		\label{fig:tp3output360}
	\end{subfigure}
	\caption{Illustration of the Wasserstein flow technique for non-parametric fitting of distributions.}
\end{figure}


\paragraph{Exercise 4.} \Cref{fig:FitEvolutionSinkhornScore} shows the evolution of the fit of the distribution using the standard Sinkhorn score
\[
	\calE(z) = W_\epsilon\left(\frac{1}{n}\sum_{i=1}^n \delta_{z_i}, \beta\right).
\]
with $20$ iterations, gradient step size $\tau = 0.07$, and varying regularization strengths 
\[
	\epsilon = 0.01, 0.02, 0.05, 0.08, 0.1, 0.2, 0.3, 0.4, 0.5, 0.6, 0.7, 0.8
\]
The score value as a function of $\epsilon$ is given \cref{fig:SinkhornScoreEpsilon}. As the regularization strength increases, the fit worsens.


\subparagraph{Sinkhorn divergence score} The Sinkhorn divergence score is given by
\[
	\calE(z) = W_\epsilon\left(\gamma(z), \beta\right) - \frac{1}{2}W_\epsilon(\gamma(z), \gamma(z))
	-\frac{1}{2}W_\epsilon(\beta, \beta)
\]
where $\gamma(z) = \frac{1}{n}\sum_{i=1}^n \delta_{z_i}$. The gradient works out as
\[
	\nabla\calE(z) = (\bfP_z^\star+\bfP_zz^{\star,T})z - \bfP^\star y
\]
where $\bfP_z^\star$ solves $W_\epsilon(\gamma(z),\gamma(z))$ and $\bfP^\star$ solves $W_\epsilon(\gamma(z),\beta)$. See \cref{fig:FitEvolutionSinkhornDivg,fig:SinkhornDivgEpsilonObj} for results: the Sinkhorn divergence has better behavior for increasing $\epsilon$.

\begin{figure}[!h]
	\centering
	\includegraphics[width=0.5\linewidth]{tp3/output_41_0}~
	\includegraphics[width=0.5\linewidth]{tp3/output_41_1}
	\includegraphics[width=0.5\linewidth]{tp3/output_41_2}~
	\includegraphics[width=0.5\linewidth]{tp3/output_41_3}
	\includegraphics[width=0.5\linewidth]{tp3/output_41_4}~
	\includegraphics[width=0.5\linewidth]{tp3/output_41_5}
	\includegraphics[width=0.5\linewidth]{tp3/output_41_6}~
	\includegraphics[width=0.5\linewidth]{tp3/output_41_7}
	\includegraphics[width=0.5\linewidth]{tp3/output_41_8}~
	\includegraphics[width=0.5\linewidth]{tp3/output_41_9}
	\includegraphics[width=0.5\linewidth]{tp3/output_41_10}~
	\includegraphics[width=0.5\linewidth]{tp3/output_41_11}
	\caption{Evolution of the fit with the standard Sinkhorn score. Regularization strength increases from left to right, top to bottom.}
	\label{fig:FitEvolutionSinkhornScore}
\end{figure}

\begin{figure}[!h]
	\centering
	\includegraphics[width=.7\linewidth]{tp3/output_42_0.png}
	\caption{Evolution of the score $\calE(z)$ after the $20$ iterations as $\epsilon$ increases.}\label{fig:SinkhornScoreEpsilon}
\end{figure}

\begin{figure}[!h]
	\centering
	\includegraphics[width=0.5\linewidth]{tp3/output_44_1}~
	\includegraphics[width=0.5\linewidth]{tp3/output_44_2}
	\includegraphics[width=0.5\linewidth]{tp3/output_44_3}~
	\includegraphics[width=0.5\linewidth]{tp3/output_44_4}
	\includegraphics[width=0.5\linewidth]{tp3/output_44_5}
	\caption{Evolution of the fit with the Sinkhorn divergence score.}
	\label{fig:FitEvolutionSinkhornDivg}
\end{figure}

\begin{figure}[!h]
	\centering
	\includegraphics[width=.7\linewidth]{tp3/output_45_0.png}
	\caption{Evolution of the Sinkhorn divergence score as $\epsilon$ increases.}\label{fig:SinkhornDivgEpsilonObj}
\end{figure}



\subsection{Generative model fitting}


\paragraph{Exercise 5.} \Cref{fig:GenerativeFitting}.
\begin{figure}[!h]
	\begin{subfigure}{.49\linewidth}
		\centering
		\includegraphics[width=0.8\linewidth]{tp3/output_49_0}
		\caption{Dataset.}
		\label{fig:output490}	
	\end{subfigure}~
	\begin{subfigure}{.49\linewidth}
		\centering
		\includegraphics[width=0.8\linewidth]{tp3/output_57_0}
		\caption{}
		\label{fig:output570}
	\end{subfigure}
	\caption{Fitting the generative model.}\label{fig:GenerativeFitting}
\end{figure}


\paragraph{Exercise 6.} We define the dataset in \cref{fig:output591}.
\begin{figure}
	\begin{subfigure}{.49\linewidth}
		\centering
		\includegraphics[width=0.7\linewidth]{tp3/output_59_1}
		\caption{Dataset}
		\label{fig:output591}
	\end{subfigure}~
	\begin{subfigure}{.49\linewidth}
		\centering
		\includegraphics[width=0.8\linewidth]{tp3/output_60_0}
		\caption{}
		\label{fig:output600}
	\end{subfigure}
	\caption{Fitting the generative model on more complex data.}\label{fig:GenerativeFittingComplex}
\end{figure}


\clearpage

\section{Semi-discrete transport}


\paragraph{Exercise 1.} We use the data defined in \cref{fig:tp4output101}.
\begin{figure}[!h]
	\begin{subfigure}[t]{.49\linewidth}
		\centering
		\includegraphics[width=0.6\linewidth]{tp4/output_10_1}
		\caption{Dataset: mixture of Gaussians and Laplace, and}
		\label{fig:tp4output101}	
	\end{subfigure}
	\begin{subfigure}[t]{.49\linewidth}
		\centering
		\includegraphics[width=0.6\linewidth]{tp4/output_18_1}
		\caption{Laguerre partition.}
		\label{fig:tp4output181}
	\end{subfigure}	
	\caption{Custom data and corresponding Laguerre cells.}
\end{figure}

\begin{figure}[!h]
	\begin{subfigure}{.49\linewidth}
		\centering
		\includegraphics[width=0.9\linewidth]{tp4/output_21_0}
		\caption{Laguerre cells throughout the gradient stepping.}
		\label{fig:tp4output210}
	\end{subfigure}~
	\begin{subfigure}{.49\linewidth}
		\centering
		\includegraphics[width=0.9\linewidth]{tp4/output_23_1}
		\caption{Evolution of the dual energy $\calE(f)$ throughout gradient ascent.}
		\label{fig:output231}
	\end{subfigure}
	\caption{Gradient ascent.}
\end{figure}




\paragraph{Exercise 2.} The stochastic optimization can be sensitive to how the learning rate $\tau_t$ is chosen. \Cref{fig:TP4StochasticOptimLargeWarmup} illustrates how increasing the warmup can lead to poor convergence.
\begin{figure}
	\begin{subfigure}{.49\linewidth}
		\centering
		\includegraphics[width=0.9\linewidth]{tp4/output_33_0}
		\caption{}
		\label{fig:output330}
	\end{subfigure}~
	\begin{subfigure}{.49\linewidth}
		\centering
		\includegraphics[width=0.9\linewidth]{tp4/output_35_1}
		\caption{}
		\label{fig:output351}
	\end{subfigure}
	\caption{Stochastic optimization with the default parameters: $\tau_t = 0.1/(1+t/10)$.}
\end{figure}

\begin{figure}
	\begin{subfigure}{.49\linewidth}
		\centering
		\includegraphics[width=0.9\linewidth]{tp4/stochasticoptim_l0_large.png}
		\caption{}
	\end{subfigure}~
	\begin{subfigure}{.49\linewidth}
		\centering
		\includegraphics[width=0.9\linewidth]{tp4/stochastic_optim_objective_l0_large.png}
		\caption{}
	\end{subfigure}
	\caption{Stochastic optimization when taking a learning rate with longer warmup $\ell_0 = 20$, so that $\tau_t = 0.1/ (1+t/20)$. Not decaying the learning rate quickly enough leads to instability and oscillations (perhaps more gradient steps would be required for convergence).}\label{fig:TP4StochasticOptimLargeWarmup}	
\end{figure}



\paragraph{Exercise 3.} \Cref{fig:OptimalQuantization} shows the result of the Lloyds algorithm on the data. The algorithm converges quickly to the criterion's optimum.
\begin{figure}
	\begin{subfigure}{.49\linewidth}
		\centering
		\includegraphics[width=0.9\linewidth]{tp4/output_43_0}
		\caption{}
		\label{fig:output430}
	\end{subfigure}~
	\begin{subfigure}{.49\linewidth}
		\centering
		\includegraphics[width=0.9\linewidth]{tp4/output_45_1}
		\caption{}
		\label{fig:output451}
	\end{subfigure}
	\caption{}\label{fig:OptimalQuantization}
\end{figure}






\end{document}
