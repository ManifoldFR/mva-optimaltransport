\documentclass[../report.tex]{subfiles}

\begin{document}

The idea of \cite{benamou:hal-01295299} is to write the MFG PDEs in variational form.
\textcite{benamou:hal-01295299} show that the MFG can be reformulated to the following variational problem:
\begin{equation}\label{eq:EulerianProblem}
\begin{aligned}
&\inf_{\rho,v} J(\rho, v) =
\frac{1}{2}\int_0^T\int_\Omega |v(t,x)|^2 \,d\rho_t(x)\,dt + \int_0^T F(\rho_t)\,dt + G(\rho_T)
\\
\suchthat\ &\partial_t \rho_t - \frac{\sigma^2}{2}\Delta \rho_t + \divg(\rho_t v) = 0
\end{aligned}
\end{equation}
where $(\rho_t)_{t\in[0,T]}$ is a trajectory in the space of measures and $v$ is a vector field on $[0,T] \times \Omega$ most likely lying in a Sobolev space.

This point of view is called \textit{Eulerian}. The nonconvex problem is solved in \textcite{benamou:hal-01295299,benamou2015lagrangian} by solving an equivalent convex problem with an augmented saddle-point approximation.


\subsection{Lagrangian formulations}

\textcite{benamou:hal-01295299,benamou2018entropy} introduce a \textit{Lagrangian} point of view: the variational problem is changed to a weighted energy minimization problem on the space of trajectories. The energy of all paths is weighted using a measure on the space of trajectories, and we seek the optimal such measure: this becomes an optimal transport problem. The first paper on variational MFG \cite{benamou:hal-01295299} considers a framework of optimizing a weighted kinetic energy $\calK(Q) = \int \int_0^T \frac{1}{2}|\dot\xi(t)|^2\,dt\,dQ(\xi)$. Equivalence between this point of view and the previous formulation is shown in that same paper.

Instead of solving for this kinetic objective (with finite difference or finite element discretization), the paper by \textcite{benamou2018entropy} propose an entropic objective to allow for a more computationally efficient numerical method adapted from the Sinkhorn algorithm \textcite{cuturi2013sinkhorn}. \textcite{benamou2018entropy} consider games on the domain $\Omega = \RR^d$ and introduce the following variational framework:


\subsubsection{The Wiener space of trajectories}

Denote $\calX = \calC([0,T], \RR^d)$ the Wiener space of agents' trajectories $[0,T] \to\RR^d$. The space $\calX$ is equipped with the Wiener measure.
In the unitary viscosity case ($\sigma = 1$), the Wiener measure $R$ on $\calX$ is defined as follows: the measure of a set of paths $\mathcal{A} \subset \calX$ is
\[
	R(\mathcal{A}) = \int_{\RR^d}\PP(x+W \in \mathcal{A})\, dx
\]
where $W$ is the standard Brownian motion (starting at $0$).
In the non-unitary viscosity case ($\sigma \neq 1$), we introduce $\greenfont\epsilon = \sigma^2$ and the Wiener measure $\greenfont R_\epsilon$ associated with Wiener processes scaled by $\sigma$.
It is an analogue in the space $\calX$ to the usual finite-dimensional Lebesgue measure\footnote{\url{https://en.wikipedia.org/wiki/Infinite-dimensional_Lebesgue_measure}}. \cites{benamou:hal-01295299,benamou2015lagrangian} This construction is needed because it allows us to talk about notions and \textbf{density} and \textbf{entropy of a probability measure on $\calX$} with respect to the Wiener measure, which is used to define the energy to minimize.


Measures $Q \in \calP(\calX)$ can also be seen as trajectories $(Q_t)_{t\in[0,T]}$ in $\calP(\RR^d)$ with
\[
Q_t = e_{t\#}Q \in \calP(\RR^d)
\]
the push-forward of $Q$ by the evaluation map $e_t\colon \xi \in\calX\longmapsto \xi(t)$. This defines a natural injection $\underline{i} \colon \calP(\calX) \rightarrow \calC([0,T], \calP(\Omega))$ from the probability measures on trajectory space to the trajectory space of probabilities on $\RR^d$. We also denote the more general $N$-marginals $Q_{t_1,\ldots,t_n} = (e_{t_1},\ldots, e_{t_n})_\# Q$ for $0\leq t_1 < \cdots < t_N \leq T$.

\paragraph{Marginals of the Wiener measure.} \textcite{benamou2018entropy} provide the following results on the Wiener measure $R$ on $\RR^d$.
\begin{itemize}
	\item The single marginals $R_t$ are the Lebesgue measure $\mathcal{L}^d$ on $\RR^d$.
	\item The 2-marginals have density on $\RR^d \times \RR^d$
	\begin{equation}\label{eq:2MarginWienerMeasure}
		R_{s,t}(x,y) = P_{t-s}(y-x)
	\end{equation}
	where $P_t$ is the standard $d$-dimensional heat kernel:
	\begin{equation}\label{eq:StandardDHeatKernel}
		P_t(u) = \exp\left(
		-\frac{|u|^2}{2t}
		\right)
	\end{equation}
	\item The $N$-marginals are given by
	\begin{equation}
		R_{t_1,\ldots,t_N}(x_1,\ldots,x_n) = 
		\prod_{i=1}^{N-1}
		P_{h}(x_{i+1}-x_i)
	\end{equation}
\end{itemize}
The last property is especially \textbf{important} for computational reasons, as we will see later. When considering non-unit viscosity and a Wiener measure $R_\epsilon$, the right-hand side of \eqref{eq:2MarginWienerMeasure} changes to $P_{(t-s)\epsilon}(y-x)$. Partial integration with respect to the 2-marginal measure $R_{0,h}$ is actually convolution with respect to the heat kernel $P_h$:
\[
	\int_\Omega u(x) R_{0,h}(x,y)\,dx =
	\int_\Omega u(x) P_h(y-x)\,dx =
	(u \star P_h)(y)
\]
The effect of integration against the $N$-marginal can then be deduced by induction, as we will see in \Cref{prop:efficientConvol}.


But what happens when we change domains? From a computational point of view $\RR^d$ cannot be represented exactly in a machine, and many interesting problems (e.g. crowd motion) involve finite subdomains $\Omega \subsetneq \RR^d$.


\subsubsection{Entropy objective}

This method introduces entropic regularization in the problem. As shown in the initial paper \cite{benamou2018entropy}, the resulting variational problem and associated numerical scheme become regularizations of the problem from \cite{benamou:hal-01295299,benamou2015lagrangian}.
More precisely, the energy to minimize is the entropy with respect to the Wiener measure:
\begin{equation}
	\mathcal{E}(Q) = \epsilon H(Q | R_\epsilon) =
	\begin{cases}
	\epsilon \int_\calX \log(dQ/dR_\epsilon)\, dQ
	& \mbox{if } Q \ll R_\epsilon  \\
	+\infty & \mbox{otherwise}
	\end{cases}
\end{equation}

The entropic Lagrangian variational formulation of the MFG \eqref{eq:QuadraticMFG} is:
\begin{equation}\label{eq:EntropyLagrangianProblem}
	\inf_{Q\in\calP(\calX)}
	\epsilon H(Q|R_{\epsilon}) + \int_0^T F(Q_t)\,dt + G(Q_T)\quad
	\suchthat\ Q_0 = \rho_0
\end{equation}


The paper by \textcite[sec.~4.2]{benamou2018entropy} shows the following equivalence property between the variational formulations of the mean-field game:
\begin{prop}[Equivalence theorem]
	Suppose the measure of trajectories $Q \in \calP(\calX)$ solves the entropic variational MFG \eqref{eq:EntropyLagrangianProblem}. Then the trajectory of measures $\rho = (Q_t)_{t\in[0,T]}$ solves the Eulerian variational MFG problem \eqref{eq:EulerianProblem}.
\end{prop}




\end{document}