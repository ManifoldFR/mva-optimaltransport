\documentclass[11pt]{article}

\usepackage{subfiles}
\usepackage[a4paper,hmargin=2.8cm]{geometry}
\usepackage{subcaption}
\usepackage{mathtools}
\usepackage{amssymb}
\usepackage{dsfont,mathrsfs}
\usepackage[dvipsnames]{xcolor}

\usepackage[
	ruled,vlined,
	linesnumbered
]{algorithm2e}

\usepackage{amsthm}
\usepackage[
	framemethod=TikZ
]{mdframed}

\usepackage{hyperref,cleveref}
\usepackage{graphicx}
\usepackage{enumitem}

\setlist{itemsep=0pt,topsep=0pt}

\usepackage{csquotes}
\usepackage[
	sorting=none,
	minnames=1,
	maxcitenames=2,
	backend=biber
]{biblatex}

\addbibresource{../bibliography/references.bib}

%% Hyperref %%

\hypersetup{
	colorlinks,
	citecolor=Green
}

\crefalias{prop}{proposition}

%%% DEFINE MACROS %%%

%% Math %%

\newcommand{\RR}{\mathbb{R}}
\newcommand{\TT}{\mathbb{T}}
\newcommand{\QQ}{\mathbb{Q}}
\newcommand{\NN}{\mathbb{N}}
\newcommand{\BB}{\mathbb{B}}
\newcommand{\WW}{\mathbb{W}}
\newcommand{\EE}{\mathbb{E}}
\newcommand{\PP}{\mathbb{P}}

\newcommand{\bfR}{\mathbf{R}}
\newcommand{\bfP}{\mathbf{P}}


\newcommand{\calC}{\mathcal{C}}
\newcommand{\calI}{\mathcal{I}}
\newcommand{\calK}{\mathcal{K}}
\newcommand{\calL}{\mathcal{L}}
\newcommand{\calP}{\mathcal{P}}
\newcommand{\calO}{\mathcal{O}}
\newcommand{\calT}{\mathcal{T}}
\newcommand{\calS}{\mathcal{S}}
\newcommand{\calM}{\mathcal{M}}
\newcommand{\calW}{\mathcal{W}}
\newcommand{\calX}{\mathcal{X}}

\newcommand{\suchthat}{\mathrm{s.t.}}

\renewcommand{\phi}{\varphi}
\renewcommand{\epsilon}{\varepsilon}

\DeclareMathOperator{\divg}{div}
\DeclareMathOperator{\Ent}{Ent}
\DeclareMathOperator{\supp}{supp}
\DeclareMathOperator*{\argmin}{argmin}
\DeclareMathOperator*{\argmax}{argmax}

\DeclareMathOperator{\KL}{KL}
\DeclareMathOperator{\proj}{proj}
\DeclareMathOperator{\prox}{prox}

\numberwithin{equation}{section}

%% Colors %%

\colorlet{lightblue}{RoyalBlue!13!white}
\colorlet{midblue}{RoyalBlue!70}
\colorlet{midgreen}{OliveGreen!65}
\colorlet{darkred}{Red!90!Black}

\newcommand{\redfont}{\color{darkred}}
\newcommand{\bluefont}{\color{RoyalBlue}}
\newcommand{\greenfont}{\color{Green!90!black}}

%% THEOREM ENVS %%

\mdfsetup{
	outerlinewidth=1pt,
	innertopmargin=0cm,
}

%\newtheorem{prop}{Proposition}
\newmdtheoremenv[
	linecolor=midblue,
	backgroundcolor=white,
]{prop}{Proposition}


\theoremstyle{definition}
\newmdtheoremenv[
	hidealllines=true,
	leftline=true,
	linecolor=Red!70
]{remark}{Remark}

\author{
	Wilson Jallet\\
	\textit{École polytechnique, ENS Paris-Saclay}
}
\title{
	{\Large\itshape\sffamily Computational Optimal Transport: Final Project}\\
	{\Large A regularized Optimal Transport formulation for variational Mean-Field Games}}

\begin{document}
\maketitle


\begin{abstract}
	Mean-field games (MFG) are strategic decision-making problems designed to approximate complex large-scale, many-agent differential games using partial differential equations and study their Nash equilibria. This gives access to the convenient theoretical tools of differential equations. In recent years, work has been done on finding variational formulations for MFGs so they can be written as convex optimization problems and eventually be connected to the theory of optimal transport \cite{benamou:hal-01295299,benamou2015lagrangian}. In particular, a recent paper by \textcite{benamou2018entropy} explores a class of games that can be written as minimal-entropy problems over a Wiener space, with an efficient numerical algorithm in tow.
\end{abstract}



\section{Quadratic Mean-Field Games}

\subsection{Control problem and Nash equilibrium}

\subfile{parts/intro.tex}

\subsection{Variational formulation}

\subfile{parts/variationalformulation.tex}


\section{Numerical algorithm}

\subfile{parts/algo.tex}



\section{Examples}

\subfile{parts/examples.tex}




\printbibliography{}







\end{document}
