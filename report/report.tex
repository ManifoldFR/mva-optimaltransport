\documentclass[11pt]{article}

\usepackage{subfiles}
\usepackage[a4paper,hmargin=2.8cm,vmargin=3.6cm]{geometry}
\usepackage{subcaption}
\usepackage{mathtools}
\usepackage{amssymb}
\usepackage{dsfont,mathrsfs}
\usepackage[dvipsnames]{xcolor}

\usepackage[
	ruled,vlined,
	linesnumbered
]{algorithm2e}

\usepackage{amsthm}
\usepackage[
	framemethod=TikZ
]{mdframed}

\usepackage{verbatim}

\usepackage{hyperref,cleveref}
\usepackage{graphicx}
\usepackage{enumitem}

\setlist{itemsep=0pt,topsep=0pt}

\usepackage{csquotes}
\usepackage[
	sorting=none,
	minnames=1,
	maxcitenames=2,
	backend=biber
]{biblatex}

\addbibresource{../bibliography/references.bib}

%% Hyperref %%

\hypersetup{
	colorlinks,
	citecolor=Green
}

\crefalias{prop}{proposition}

%%% DEFINE MACROS %%%

%% Math %%

\newcommand{\RR}{\mathbb{R}}
\newcommand{\TT}{\mathbb{T}}
\newcommand{\QQ}{\mathbb{Q}}
\newcommand{\NN}{\mathbb{N}}
\newcommand{\BB}{\mathbb{B}}
\newcommand{\WW}{\mathbb{W}}
\newcommand{\EE}{\mathbb{E}}
\newcommand{\PP}{\mathbb{P}}

\newcommand{\bfR}{\mathbf{R}}
\newcommand{\bfP}{\mathbf{P}}


\newcommand{\calC}{\mathcal{C}}
\newcommand{\calE}{\mathcal{E}}
\newcommand{\calI}{\mathcal{I}}
\newcommand{\calH}{\mathcal{H}}
\newcommand{\calK}{\mathcal{K}}
\newcommand{\calL}{\mathcal{L}}
\newcommand{\calP}{\mathcal{P}}
\newcommand{\calO}{\mathcal{O}}
\newcommand{\calT}{\mathcal{T}}
\newcommand{\calS}{\mathcal{S}}
\newcommand{\calM}{\mathcal{M}}
\newcommand{\calW}{\mathcal{W}}
\newcommand{\calX}{\mathcal{X}}

\newcommand{\suchthat}{\mathrm{s.t.}}

\renewcommand{\phi}{\varphi}
\renewcommand{\epsilon}{\varepsilon}

\DeclareMathOperator{\divg}{div}
\DeclareMathOperator{\Ent}{Ent}
\DeclareMathOperator{\supp}{supp}
\DeclareMathOperator*{\argmin}{argmin}
\DeclareMathOperator*{\argmax}{argmax}

\DeclareMathOperator{\KL}{KL}
\DeclareMathOperator{\proj}{proj}
\DeclareMathOperator{\prox}{prox}

\numberwithin{equation}{section}

%% Colors %%

\colorlet{lightblue}{RoyalBlue!13!white}
\colorlet{midblue}{RoyalBlue!70}
\colorlet{midgreen}{OliveGreen!65}
\colorlet{darkred}{Red!90!Black}

\newcommand{\redfont}{\color{darkred}}
\newcommand{\bluefont}{\color{RoyalBlue}}
\newcommand{\greenfont}{\color{Green!90!black}}

%% THEOREM ENVS %%

\mdfsetup{
	outerlinewidth=1pt,
	innertopmargin=0cm,
}

%\newtheorem{prop}{Proposition}
\newmdtheoremenv[
	hidealllines=true,
	leftline=true,
	linecolor=midblue,
	backgroundcolor=RoyalBlue!3,
]{prop}{Proposition}


\theoremstyle{definition}
\newmdtheoremenv[
	hidealllines=true,
	leftline=true,
	linecolor=Red!70,
]{remark}{Remark}

%% TITLE, AUTHOR %%

\author{
	Wilson Jallet\\
	\textit{École polytechnique, ENS Paris-Saclay}
}
\title{
	{\Large\sffamily Computational Optimal Transport -- Project report}\\
	{\Large A regularized Optimal Transport formulation for variational Mean-Field Games}}

\begin{document}
\maketitle


\begin{abstract}
	Mean-field games (MFG) are strategic decision-making problems designed to approach complex large-scale, many-agent differential games using partial differential equations and study their Nash equilibria using the convenient theoretical tools of differential equations. In recent years, work has been done on finding variational formulations for MFGs so they can be written as convex optimization problems and eventually be connected to the theory of optimal transport \cite{benamou:hal-01295299,benamou2015lagrangian}.
	A paper by \textcite{benamou2018entropy} explores a class of variational MFGs that can be written as penalized minimal-entropy problems over a Wiener space, with an efficient numerical algorithm in tow.
	
	In this report, we summarize the general framework of mean-field games, the ideas behind variational formulations for MFG and how they connect to optimal transport. Then, we clarify the algorithms introduced by \cite{benamou2018entropy}, provide a full implementation, analyze numerical results and discuss theoretical and practical limitations of the approach and possible extensions.
\end{abstract}



\section{Variational Mean-Field Games}

\subsection{Control problem and Nash equilibrium}

\subfile{parts/intro.tex}

\subsection{Variational formulation}

\subfile{parts/variationalformulation.tex}


\section{Numerical algorithm}

\subfile{parts/algo.tex}



\section{Examples}\label{sec:Examples}

\subfile{parts/examples.tex}


\section{Conclusion and further work}

The variational mean-field game framework and its OT regularization introduced by \textcite{benamou:hal-01295299,benamou2018entropy} provide useful computational approaches: it shows that some MFGs can be written as optimization problems, and approached by entropic transport problems that can be solved quickly using Sinkhorn scaling.

However, the entropic transport point of view has some \textbf{limitations} so far: entropy minimization approaches for optimal time problems (see \cref{rem:SmartPotential}), ergodic problems and non-quadratic Hamiltonians have not been derived yet. Lagrangian formulations of these problems connecting them to OT are required.

\paragraph{Further work: Domain topology and heat kernel}\label{sec:ExtensionTopologyHeatKernel} Extensions to non-convex domains and manifolds might be necessary to handle cases where imposing zero-mass constraints are insufficient and still produce non-physical behavior (as in the crowd motion example). 
For complicated domains such as the second room \cref{fig:Room3} we get non-physical behavior where the agents pass through walls. The domain $\Omega$ in that example is suppoed to contain the obstacles and it is non-convex: obviously the Euclidean heat kernel $P_t$ is \textit{not} the exact 2-marginal of the Wiener measure, which justifies looking at using a more appropriate kernel. As a first step, we propose the following approach where we change out heat kernels $\bfP$.

We take inspiration from \textcite{peyr2015entropic}, which discusses an OT approach to solving PDEs using JKO flows, eventually on non-convex subdomains of $\RR^d$ and Riemannian manifolds using a \textit{geodesic distance} as a ground cost. The distance kernel $\xi(x,y) = \exp(-d_\calM(x,y)/\tau)$ is approached by solving the heat equation with a discrete Laplacian operator.
We have implemented the approach, leading to the results in \cref{fig:GeodesicKernelRoom3Example} when applied to the complicated second room with $N=10$ time steps and grid size $M=101\times 101$. The CPU time was $30$ seconds and the relative $L^\infty$ constraint error was $\sim 10^{-4}$. We obtain qualitatively much better results when compared with the Euclidean heat kernel \cref{fig:NMargEx3}.

Further work should be done in this direction: \textcite{benamou2018entropy} could be extended with results generalized to manifolds.


\begin{figure}[h]
	\centering
	\begin{subfigure}[c]{.4\linewidth}
		\includegraphics[width=\linewidth]{../project/images/geodesic_room3/congestion_plot.pdf}
		\caption{Congestion plot for room 3 with the geodesic kernel.}
	\end{subfigure}~
	\begin{subfigure}[c]{.4\linewidth}
		\includegraphics[width=\linewidth]{../project/images/geodesic_room3/hilbert_convergence.png}
		\caption{Convergence of the Hilbert metric.}
	\end{subfigure}
	\begin{subfigure}[c]{.8\linewidth}
		\includegraphics[width=\linewidth]{../project/images/geodesic_room3/transport.pdf}
		\caption{MFG evolution using the geodesic heat kernel.}
	\end{subfigure}
	\caption{Solution of the crowd motion MFG in the more complex room 2 using the geodesic heat kernel. Hilbert convergence threshold $\eta = 10^{-3}$.}\label{fig:GeodesicKernelRoom3Example}
\end{figure}



\printbibliography{}







\end{document}
